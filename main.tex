% Copyright 2004 by Till Tantau <tantau@users.sourceforge.net>.
%
% In principle, this file can be redistributed and/or modified under
% the terms of the GNU Public License, version 2.
%
% However, this file is supposed to be a template to be modified
% for your own needs. For this reason, if you use this file as a
% template and not specifically distribute it as part of a another
% package/program, I grant the extra permission to freely copy and
% modify this file as you see fit and even to delete this copyright
% notice. 

\documentclass{beamer}
% There are many different themes available for Beamer. A comprehensive
% list with examples is given here:
% http://deic.uab.es/~iblanes/beamer_gallery/index_by_theme.html
% You can uncomment the themes below if you would like to use a different
% one:
%\usetheme{AnnArbor}
%\usetheme{Antibes}
%\usetheme{Bergen}
%\usetheme{Berkeley}
%\usetheme{Berlin}
%\usetheme{Boadilla}
%\usetheme{boxes}
%\usetheme{CambridgeUS}
%\usetheme{Copenhagen}
%\usetheme{Darmstadt}
%\usetheme{default}
%\usetheme{Frankfurt}
%\usetheme{Goettingen}
%\usetheme{Hannover}
%\usetheme{Ilmenau}
%\usetheme{JuanLesPins}
%\usetheme{Luebeck}
\usetheme{Madrid}
%\usetheme{Malmoe}
%\usetheme{Marburg}
%\usetheme{Montpellier}
%\usetheme{PaloAlto}
%\usetheme{Pittsburgh}
%\usetheme{Rochester}
%\usetheme{Singapore}
%\usetheme{Szeged}
%\usetheme{Warsaw}
\graphicspath{{figures/}}
\definecolor{cwired}{rgb}{0.803,0.0,0.227}
\setbeamercolor{structure}{fg=cwired}
\title{Inverse Problems in Plasma Diffusion}

% A subtitle is optional and this may be deleted
\subtitle{A Probabilistic Approach for Physical Systems}

\author{Mandar ~Chandorkar}
% - Give the names in the same order as the appear in the paper.
% - Use the \inst{?} command only if the authors have different
%   affiliation.

\institute[CWI \& INRIA] % (optional, but mostly needed)
{
  \inst{1}%
  Multiscale Dynamics\\
  CWI, Amsterdam
  \and
  \inst{2}%
  TAO Research Unit\\
  INRIA, CNRS, Saclay}
% - Use the \inst command only if there are several affiliations.
% - Keep it simple, no one is interested in your street address.

\date{19 Sept 2017}
% - Either use conference name or its abbreviation.
% - Not really informative to the audience, more for people (including
%   yourself) who are reading the slides online

\subject{Machine Learning}
% This is only inserted into the PDF information catalog. Can be left
% out. 

% If you have a file called "university-logo-filename.xxx", where xxx
% is a graphic format that can be processed by latex or pdflatex,
% resp., then you can add a logo as follows:

\pgfdeclareimage[height=0.5cm]{university-logo}{figures/cwi-logo}
\logo{\pgfuseimage{university-logo}}

% Delete this, if you do not want the table of contents to pop up at
% the beginning of each subsection:
\AtBeginSubsection[]
{
  \begin{frame}<beamer>{Outline}
    \tableofcontents[currentsection,currentsubsection]
  \end{frame}
}

% Let's get started
\begin{document}

\begin{frame}
  \titlepage
\end{frame}

\begin{frame}{Outline}
  \tableofcontents
  % You might wish to add the option [pausesections]
\end{frame}

% Section and subsections will appear in the presentation overview
% and table of contents.
\section{Problem}

\subsection{Magnetosphere}

\begin{frame}
  \begin{figure}[h]
        \includegraphics[width=0.8\textwidth]{Magnetosphere_rendition}
        \caption{Sun-Magnetosphere System}
        \label{fig:Solar}
      \end{figure}
\end{frame}

\begin{frame}{Radiation Belts}
  \textbf{Van Allen radiation belts} are a layer of trapped charged particles around the Earth.
  \begin{itemize}
  \item {
      \textbf{Outer radiation belt}, mainly composed of high energy electrons (energetic range around 0.1 to 10 MeV)
  }
  \item {
      \textbf{Inner radiation belt}, composed of high concentrations of electrons (energetic range of hundreds of keV) and energetic protons.
  }
  \end{itemize}
\end{frame}

\subsection{Plasma Dynamics:  Adiabatic Theory}

% You can reveal the parts of a slide one at a time
% with the \pause command:
\begin{frame}{Plasma Motions}

  Motion of charged particles is decomposed into three components.
  
   \begin{figure}[h]
        \includegraphics[width=0.5\textwidth]{adiabatic_motions}
        \caption{Adiabatic Motions}
        \label{fig:Adiabatic}
      \end{figure}
\end{frame}

\begin{frame}{Adiabatic Invariants}
  For each component of plasma motion, there is one \emph{adiabatic
    invariant} which is assumed to be conserved in plasma motion.

  \begin{itemize}
  \item<1->{
      Cyclotron Motion: $\mathcal{M} = m_{0} v^{2}_{\perp}/2B =
      \frac{m_{0}v^2 sin^{2}\alpha}{2B}$
    }
  \item<2->{
      Bounce Motion: $J = \int{m_0 v_{\parallel}ds}$
    }
  \item<3->{
      Drift Motion: $\Phi = \int{\int{\mathbf{B} d\mathbf{S}}}$
    }
  \end{itemize}

\end{frame}

\subsection{Plasma Dynamics: Diffusion Theory}

\begin{frame}{Violation of Invariants}
  \begin{itemize}
  \item<1->{Radial Diffusion: Affects only third invariant $\Phi$}
  \item<2->{Pitch Angle Diffusion: Affects all three, leads to loss of
    particles from radiation belt}
  \end{itemize}
\end{frame}

\begin{frame}{Phase Space Density}
  Quantity of interest in plasma diffusion is the so called
  \emph{phase space density} $f(\mathcal{M}, J, \Phi, t)$.
\end{frame}

\begin{frame}{Fokker-Plank Equation}
  Canonical form of plasma diffusion \cite{Schulz1974} 
  \begin{align*}
    \frac{\partial{f}}{\partial{t}} &= \sum^{3}_{p,q = 1}
    \frac{\partial}{\partial{J_{p}}} \left( \kappa_{pq}
    \frac{\partial{f}}{\partial{J_{q}}} \right) \\
    J_1 &= \mathcal{M} \\
    J_2 &= J \\
    J_{3} &= \Phi
  \end{align*}
\end{frame}

\begin{frame}{Simplifying Fokker-Plank}

  \begin{itemize}
  \item Assume Dipole magnetic field
  \item Substitute $\ell = 2 \pi B_{E} R^{2}_{E}/\Phi$, a coordinate
    which monotonically increases as one goes farther from the Earth.
  \item Approximate pitch-angle diffusion (particle loss) as
    $\lambda(\ell, t) f$.
  \end{itemize}
\end{frame}

\begin{frame}{Radial Diffusion}
  The simplified system \cite{Walt1970}, now looks like
  \begin{equation*}
    \frac{\partial{f}}{\partial{t}} = \ell^2 \frac{\partial}{\partial{\ell}} \left( \frac{\kappa(\ell,
        t)}{\ell^{2}} \frac{\partial{f}}{\partial{\ell}}
    \right)_{\mathcal{M}, J} - \lambda(\ell,
    t) f
  \end{equation*}

  
\end{frame}

\begin{frame}{Key Quantities}
  \begin{itemize}
  \item {
      $f$: Plasma \emph{phase space density}
    \pause % The slide will pause after showing the first item
  }
  \item {   
      $\kappa(\ell, t)$: Diffusion field.
  }
  % You can also specify when the content should appear
  % by using <n->:
  \item<3-> {
    $\lambda(\ell, t)$: Particle loss rate.
  }
  \end{itemize}
\end{frame}

\subsection{Parameter Estimation}

\begin{frame}{Inverse Problem}
  \begin{block}{Data}
    Given a set of (noisy) observations
    \begin{align*}
      & \mathcal{D} = \{ ((\ell_1, t_1), y_1), ((\ell_2, t_2), y_2),
                    \cdots, ((\ell_n, t_n), y_n) \} \\
      & y(\ell, t) = f(\ell, t) + \epsilon \\
      & \epsilon \sim \mathcal{N}(0, \sigma^2)
    \end{align*}
  \end{block}
  
  \begin{block}{Inference}
    The electron loss rate $\lambda(\ell, t)$
  \end{block}
\end{frame}

\begin{frame}{Challenges}
  \begin{itemize}
  \item<1->{
      Data sparsity \& irregularity.
    }
  \item<2->{Non-identifiable and Ill-posed}
  \item<3->{Combine sparse data with reduced physics}
  \end{itemize}
\end{frame}


\section{Methodology}

\subsection{Scheme}
\begin{frame}
  \begin{itemize}
  \item<1->{
      Parameterize $\kappa(\ell, t), \lambda(\ell, t)$
    }
  \item<2->{Fit a surrogate for the physical model}
  \item<3->{Quantify likelihood of observed data conditioned on
      parameters of $\lambda$}
  \item<4->{Perform MCMC based sampling}
  \end{itemize}
\end{frame}

\subsection{Parameterization}


\begin{frame}{$\kappa$ \& $\lambda$}
  Space weather literature outlines a way to parameterize radial
  diffusion unknowns \cite{JGRA:JGRA15067}.
  \begin{equation*}
    \kappa(\ell, t), \lambda(\ell, t) \sim \alpha \ell^{\beta} 10^{b Kp(t)}
  \end{equation*}

  The parameters of $\kappa$ are set to values already used in
  literature, we focus on inference of $\lambda$ parameters only.
\end{frame}

\subsection{Model Surrogate}

\begin{frame}{Radial Basis Functions}
  \begin{block}{Basis Expansion}
    The approximation to $f(\ell, t)$ is constructed as a radial basis expansion.
    \begin{align*}
      x &= (\ell, t)\\
      \hat{f}(x) &= \sum_{i}{w_{i} \phi(||x - x_{i}||/\rho_{i})}
    \end{align*}
  \end{block}

  \begin{block}{Goodness of fit}
    The coefficients $w_i$ are calculated by optimizing squared error
    with respect to observed data and \emph{ghost points} where
    deviations from radial diffusion dynamics are penalized.
    \begin{equation*}
      \mathcal{J}(w | \theta, \nu, x_{i}, f_{i}, g_{j}) = \frac{1}{2}\sum_{i}{|\hat{f}(x_{i}) -
        y_{i}|^2} + \frac{1}{2} \nu \sum_{j}{|\mathcal{L}_{\theta}[\hat{f}(x_{j})]|^2}
    \end{equation*}
  \end{block}
\end{frame}

\subsection{Bayesian Inference}

\begin{frame}{Likelihood}
  Multivariate normal distribution is used to specify the likelihood.

  \begin{align*}
    y_{1}, \cdots, y_{n} | \theta\{\lambda\}  &\sim \mathcal{N} \left( \begin{pmatrix}
        \hat{f}_{\theta}(x_{1})\\ 
        \vdots\\ 
        \hat{f}_{\theta}(x_{n})
      \end{pmatrix},  \begin{bmatrix}
        C(x_{1}, x_{1}) & \cdots & C(x_{1}, x_{n})\\ 
        \vdots & \ddots  & \vdots \\ 
        C(x_{n}, x_{1}) & \cdots & C(x_{n}, x_{n}) 
      \end{bmatrix}\right) \\\\
    C(x, y) &= s^{2} exp \left ( -\frac{||x-y||^2}{2u^2} \right )
  \end{align*}
  
\end{frame}

\section{Experiments}
\subsection{Synthetic Data}

\begin{frame}
  Synthetic phase space density data is generated using a discretized
  solver and assumed values of unknown parameters.
\end{frame}

\begin{frame}
  \begin{figure}[h]
  \includegraphics[width=\textwidth]{kp_profile.png}
   \caption{Kp Index}
  \label{fig:Kp}
\end{figure}
\end{frame}

\begin{frame}
  \begin{figure}[h]
  \includegraphics[width=\textwidth]{initial_psd.png}
  \caption{Initial Condition $f(\ell, 0)$}
  \label{fig:initialPSD}
\end{figure}
\end{frame}

\begin{frame}{Generated profiles $f(\ell, t)$}
  \begin{figure}[h]
  \includegraphics[width=\textwidth]{psdl_ex1.png}
\end{figure}
\end{frame}

\begin{frame}{Generated profiles $f(\ell, t)$}
  \begin{figure}[h]
  \includegraphics[width=\textwidth]{psdt_ex1.png}
\end{figure}
\end{frame}


\subsection{Preliminary Results}

\begin{frame}
  \begin{figure}[h]
  \includegraphics[width=\textwidth]{samples1_ex1.png}
  %\caption{Results Experiment 1}
  \label{fig:Samples1Ex1}
\end{figure}
\end{frame}

\begin{frame}
\begin{figure}[h]
  \includegraphics[width=\textwidth]{samples2_ex1.png}
   %\caption{Results Experiment 1}
  \label{fig:Samples2Ex1}
\end{figure}
\end{frame}

\begin{frame}
  \begin{figure}[h]
  \includegraphics[width=\textwidth]{hist_ex1.png}
   %\caption{Results Experiment 1}
  \label{fig:HistEx1}
\end{figure}
\end{frame}



% Placing a * after \section means it will not show in the
% outline or table of contents.
\section*{Summary}

\begin{frame}{Summary}
  \begin{itemize}
  \item
    \alert{Uncertainty} creeps into even deterministic physical systems.
  \item
    Need to intelligently combine \alert{Bayesian statistics} with
    existing phyiscal models.
  \end{itemize}
  
  \begin{itemize}
  \item
    Outlook
    \begin{itemize}
    \item
      Decrease vagueness of posterior distribution by incorporating
      boundary/initial conditions.
    \end{itemize}
  \end{itemize}
\end{frame}



% All of the following is optional and typically not needed. 
\appendix
\section<presentation>*{\appendixname}
\subsection<presentation>*{For Further Reading}

\begin{frame}[allowframebreaks]
  \frametitle<presentation>{References}
        \bibliographystyle{apalike}
        \bibliography{references}
\end{frame}


\end{document}


